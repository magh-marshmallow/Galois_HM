\documentclass{article}
\usepackage[T1]{fontenc}
\usepackage{titling}
\usepackage{babel}
\usepackage{lmodern}
\usepackage{graphicx}
\usepackage{csquotes}
\usepackage{amsmath}
\usepackage{amssymb}
\usepackage{xcolor}
\usepackage{amsfonts}
\newcommand\mypound{\scalebox{0.8}{\raisebox{0.4ex}{\#}}}

\setlength{\droptitle}{-11em}

\title{\textbf{Homework Problem Set \mypound 3}}


\author{Harvey Mudd College}
\date{January 2020}

\begin{document}

\maketitle

\noindent\large{}Let $R$ be a ring with identity $1\neq0,$ let F be a field, and let x be an indeterminate over F.
\color{black}
\section*{\hspace{-1cm}9.1.7\Large{} \color{black} \normalfont Prove that a polynomial ring in more than one variable over $R$ is not a Principal Ideal Domain.}

\newpage

\section*{\hspace{-1cm}9.2.1\Large{} \color{black} \normalfont Let $f(x) \in F[x]$ be a polynomial of degree $n \geq 1$ and let bars denote passage to the quotient $F[x]/f(x)$. Prove that for each $\overline{g(x)}$ there is a unique polynomial $g_0(x)$ of degree  $\leq n - 1$ such that $\overline{g(x)} = \overline{g_0(x)}$ (equivalently, the elements $\overline{1}, \overline{x},...,\overline{x^{n-1}}$
are a $basis$ of the vector space $F[x]/((f(x))$ in particular, the dimension of this space is $n$). [Use the Division Algorithm.].}
\newpage
\section*{\hspace{-1cm}9.2.2\Large{} \color{black} \normalfont Let $F$ be a finite field of order $q$ and let $f(x)$ be a polynomial in $F[x]$ of degree $n \geq 1$. Prove that $F[x]/(f(x))$ has $q^n$ elements. [Use the preceding exercise.]}
\newpage
\section*{\hspace{-1cm}9.2.3\Large{} \color{black} \normalfont Let $f(x)$ be a polynomial in $F[x]$. Prove that $F[x]/(f(x))$ is a field if and only if $f(x)$ is irreducible. [Use Proposition 7, Section 8.2.]}
\newpage
\color{black}
\section*{\hspace{-1cm}9.3.3\Large{} \color{black} \normalfont Let $F$ be a field. Prove that the set $R$ of polynomials in $F[x]$ whose coefficient of x is
equal to 0 is a subring of $F[x]$ and that $R$ is not a U.F.D. 
\vspace{0.3cm}
\\ $[$Show that $x^6 = {(x^2)}^3={(x^3)}^2$ gives two distinct factorizations of $x^6$ into irreducibles.$]$}
\end{document}